\documentclass[11pt,letterpaper]{article}
\usepackage{comment} % enables the use of multi-line comments (\ifx \fi) 
%
\usepackage[letterpaper,margin=0.75in]{geometry}
\usepackage{fullpage} % changes the margin
\usepackage{fancyhdr} % for footer
\usepackage[UKenglish]{isodate}% http://ctan.org/pkg/isodate for date format
%\usepackage{epigrafica}%changes default font to epigrafica
\usepackage{hyperref}%for URLs
\usepackage[]{natbib}
\newcommand*{\doi}[1]{\href{http://dx.doi.org/#1}{doi: #1}}% links DOI

\pagestyle{fancy}
\renewcommand{\headrulewidth}{0pt}

\lhead{}
\chead{}
\rhead{}
\lfoot{ENT 532 (Fall 2021) - Penn State}
\cfoot{}
\rfoot{\thepage}
\renewcommand{\footrulewidth}{0.4pt}
\title{Introduction to arthropod biodiversity, natural history, and course logistics}
\author{}

\begin{document}
\cleanlookdateon %removed ordinal date
\maketitle
\thispagestyle{fancy}
\section*{Introduction}
We will discuss systematics as a field of research, and we'll touch on the importance natural history and of specimens as data, references, and vouchers \citep{vouchers}. Read the course syllabus and familiarize yourself with the grading policy, dates of exams and due dates for assignments, \textit{etc}. We'll also discuss some ethical issues regarding the collection of organisms, and we'll get started on the collection exercise---Discover your inner Darwin.

\section*{Test yourself}
In each unit handout you will find questions. Many of these questions do not have right answers but rather are intended to stimulate creative, scientific thinking. Some of these questions could appear on an exam or lab practical, so it's worth making an effort to understand them!\\

\begin{enumerate}
\item Recall our discussion of \cite{Eisner1964} and describe two salient points. Do you know which paper you're leading a discussion of?
\item How would you describe ``natural history'', and how can understanding an organism's natural history inform our decisions to control, conserve, or otherwise interact with that organism? Describe an example.
\item Could you explain to a friend how this class is graded?
\item Can you describe three ethical issues we need to consider when collecting organisms? If you were to edit the Insect Collector's Code \citep{oath} how would you change it?
\item Could you teach a friend how to keep a field journal?
\end{enumerate}


\clearpage
\section*{Epilogue}
This handout is part of an open curriculum, initially developed by Andrew R. Deans at the Pennsylvania State University. Original files are available free for anyone to download, copy, modify, and improve at the Open Entomology GitHub repository \citep{ENT532}, which also provides a mechanism for reporting problems and other feedback:\\
\url{https://github.com/OpenEntomology/InsectBiodiversityEvolution/issues}

% adding bibliography here
\bibliographystyle{myplainnat}
\bibliography{bib}

\end{document}
